\chapter{相关工作}
\thispagestyle{fancy}
本章主要介绍系统中使用到的相关技术框架:Google App Engine,Hadoop 和 jQuery。

\section{Google App Engine}
Google App Engine\cite{gae}(GAE) 是一个开发、托管网络应用程序的平台,使用Google管理的数据中心。在使用 Google App Engine 时,开发者只需上传应用程序,不需要维护任何服务器。
Google App Engine 包含以下功能:
\begin{itemize}
\item 提供动态网络服务,完全支持常用的网络技术;
\item 提供 Datastore 持久存储,并支持查询、排序和事务;
\item 提供自动扩展和负载平衡;
\item 提供功能完善的本地开发环境,用于在开发者的计算机上模拟 GAE;
\item 应用程序可以在 Java 环境和 Python 环境中运行,每种环境提供了标准协议和常用技术以进行网络应用程序开发。
\end{itemize}
Google App Engine 中数据存储的单位是对象(或实体),实体属于一个种类,具有一组属性。查询可以检索给定种类的实体,按属性值过滤和排序。Google App Engine 提供了两个不同的存储选项,两者的区别在于它们的可用性和一致性保证:
\begin{itemize}
\item 主/从数据存储区。使用主-从复制系统,该系统在将数据写入物理数据中心时异步复制数据,为所有读取和查询都提供了强一致性,其代价是在数据中心出现问题时或进行计划内停机期间,会出现暂时的不可用性。
\item High Replication 数据存储区。使用基于 Paxos 算法的系统在各个数据中心之间复制数据。High Replication 针对读取和写入提供了非常高的可用性,其代价是写入时的延迟时间较长。大多数查询是最终一致的。
\end{itemize}

\section{Hadoop}
Hadoop\cite{hadoop12} 是一个支持分布式数据密集型计算的软件框架。它主要由分布式存储系统 HDFS 和 分布式计算框架 MapReduce 构成。

\subsection{HDFS}
HDFS\cite{hdfs10} 是一个分布式文件系统,通常由一个存储元数据的 namenode 和多个存储实际数据的 datanode 组成。HDFS 将文件分成固定大小的数据块(默认为 64MB),存储在多台机器上。HDFS 通过给数据块存多个备份(默认为 3,2 个在同一机架上,1 个在不同机架上)实现数据的持久性。HDFS 适合于对大文件的批处理操作,而不适用于低延时的用户交互式应用。

\subsection{MapReduce}
MapReduce\cite{mr04} 是 Google 提出的计算框架,用于大规模数据集的并行计算。MapReduce 作业分为 Map 和 Reduce 两个步骤,开发者只需编写 Map 和 Reduce 函数,专注于算法逻辑,系统会负责资源的管理,任务的分配,提供容错,处理失败。 Map 和 Reduce 在键值对结构的数据上操作。Map 函数并行处理输入一个数据域中的键值对,输出另一个数据域的键值对列表。
\begin{equation}
(k_1, v_1) \rightarrow list(k_2, v_2)
\end{equation}
系统根据键对列表进行分组(每个键一个组),并将它们分发到不同的机器。Reduce 函数并行处理每组数据,输出一个数据集合。
\begin{equation}
(k_2, list(v_2)) \rightarrow list(v_3)
\end{equation}

\section{jQuery}
jQuery\cite{jquery} 是一个用于简化客户端对 HTML 进行脚本操作的 JavaScript\cite{jsgood}库。借助 jQuery,开发者可以方便地操纵 HTML 文档,选择 DOM 元素,处理事件,开发 Ajax\cite{ajax}应用。而基于 jQuery 的 jQuery UI\cite{jquery-ui}使得 Web 界面的制作更加高效。jQuery 还提供插件,开发者可以将底层的交互,动画等动作抽象成模块化的控件,从而构建出强大的动态网页和网络应用。此外,jQuery 还提供了多浏览器的支持。

