\chapter{总结和展望}
\thispagestyle{fancy}
本文介绍了 VisualMiner,一个新浪微博热点话题的可视化查询系统。系统通过爬取新浪微博数据,对一个话题在时间,地域和情绪等维度上进行分组聚合,将结果数据 以可视化的方式呈现给用户。用户可以直观地了解话题的讨论数随时间的变化趋势,在地域上的分布情况以及话题中的情绪特征。

虽然如此,系统在架构和功能等方面均有很大的改进空间。

在架构方面,虽然使用 MapReduce 框架编写程序可以方便地对海量的微博数据进行并行计算,但是直接用 SQL 语言表达更加直观高效。另一方面,MapReduce 适合于离线分析,不适合于即时的用户交互,这限制了系统的功能。我们考虑在下一步尝试 Google 的 BigQuery\cite{bigquery},它是 Google 的列存储系统 Dremel\cite{dremel} 的公开版本。BigQuery 支持类 SQL 的查询语言,支持嵌套的数据格式(如 JSON),可以即时分析海量数据,这些特性很符合我们的需求。此外,它还支持将 Google App Engine 上存储的数据导入 BigQuery,方便了数据的迁移。

在功能方面,目前我们预先定义了话题,对抓取的微博数据进行离线分析。在下一步的工作中,我们希望能够定时(如每天)抓取新浪微博上提及热点话题的数据,由用户输入话题,使用 BigQuery 在线分析微博数据,用可视化技术实时地向用户呈现一个话题近期的变化趋势(如与前一天相比的变化量),或是当前各地域该话题的讨论情况。此外,基于关键词的情绪分析十分粗糙(例如 “不快乐” 虽然包含 
“快乐” 一词,表达的却是完全相反的情绪),我们希望能够对微博进行语义分析,挖掘其中的情绪特征。

我们希望在未来的工作中在这些方面继续探索,不断完善我们的系统。




